%  Updated: May 29, 1995 at 13:00

\documentstyle[12pt]{article}

\textwidth=5.6in
\textheight=8.1in
\parskip=0pt  %\bigskipamount
\oddsidemargin=.5in
\topmargin=-.17in

\def\C{{\rm\kern.24em \vrule width.02em height1.4ex depth-.05ex \kern-.26em
C}}
\def\R{{\rm I\kern-.20em R}}
\def\Z{{\rm\kern.26em \vrule width.02em height0.5ex depth0ex \kern.04em
\vrule  width.02em height1.47ex depth-1ex \kern-.34em Z}}
\def\N{{\rm I\kern-.20em N}}
\def\Q{{\rm\kern.24em \vrule width.02em height1.4ex depth-.05ex \kern-.26em
Q}}

\begin{document}
\pagenumbering{roman}

\begin{titlepage}

\begin{center}
{\Large The Computer Calculation of Lie Point Symmetries} \\
{\Large of Large Systems of Differential Equations}  \\[.25in] 
B. Champagne$^{\dagger}$\footnote[1]{INRS-T\'{e}l\'{e}communications, 
Universit\'{e} du Qu\'{e}bec, 3 Place du Commerce, Verdun, Qu\'{e}bec,
Canada H3E 1H6}, 
W. Hereman$^{\ddagger}$\footnote[2]{This work was partly supported by AFOSR
under Grant No. 85-NM-0263.}\\
\& P. Winternitz$^{\dagger}$\footnote[3]{Ce rapport a \'{e}t\'{e} publi\'{e} 
gr\^{a}ce \`{a} une subvention
du fonds FCAR pour l' aide et le 
soutien \`{a} la recherche.} \\[.25in]

$\dagger$ Centre de Recherches Math\'{e}matiques \\
Universit\'{e} de Montr\'{e}al \\
C.P. 6128, Succursale A \\
Montr\'{e}al, Qu\'{e}bec, Canada H3C 3J7 \\ [.25in]

$\ddagger$Department of Mathematical and Computer Sciences  \\
Colorado School of Mines \\
Golden, CO 80401, USA \\[.60in]

{\em \today}\\
\end{center}
\vskip 5pt

\end{titlepage}

\leftline{\Large {\bf Contents}}
\vskip 2pt
\contentsline {section}{\numberline {}Abstract}{iii}
\contentsline {section}{\numberline {}Program Summary}{iv}
\contentsline {section}{\numberline {}Long Write-Up}{1}
\contentsline {section}{\numberline {1}Introduction}{1}
\contentsline {subsection}{\numberline {1.1}The problem}{1}
\contentsline {subsection}{\numberline {1.2}Review of symbolic programs}{2}
\contentsline {subsection}{\numberline {1.3}The program symmgrp.max}{4}
\contentsline {section}{\numberline {2}Procedure for Computing the 
Determining Equations}{6}
\contentsline {section}{\numberline {3}Description of the Program}{8}
\contentsline {subsection}{\numberline {3.1}Description of the principal 
functions}{9}
\contentsline {subsection}{\numberline {3.2}List of principal identifiers}{11}
\contentsline {subsection}{\numberline {3.3}Input data}{11}
\contentsline {subsection}{\numberline {3.4}Output data}{13}
\contentsline {subsection}{\numberline {3.5}Type of systems}{13}
\contentsline {section}{\numberline {4}Example: The Karpman Equations}{15}
\contentsline {section}{\numberline {5}Conclusion}{22}
\contentsline {section}{\numberline {}Notes added in proof}{22}
\contentsline {section}{\numberline {}Acknowledgements}{24}
\contentsline {section}{\numberline {}References}{25}
\contentsline {section}{\numberline {}Test Run Output}{29}
\contentsline {section}{\numberline {}Listing of symmgrp.max }{35}
\contentsline {section}{\numberline {}Listing of printeqn.max }{47}

\vfill
\newpage

\begin{abstract}

A MACSYMA program is presented that greatly helps in the calculation of
Lie symmetry groups of large systems of differential equations. 

The program calculates the determining equations
for systems of $m$ differential equations of order $k$, 
with $p$ independent and $q$ dependent variables, where $m, k, p $ and
$q$ are arbitrary positive integers. 

The program automatically produces a 
list of determining equations for the coefficients of the vector field.
This list has been parsed so that it is
free of duplicate equations and trivial differential redundancies.
Numerical factors and non zero parameters occuring as factors are also
removed.  

From the solution of these determining equations one can construct
the Lie symmetry group. 

An example shows the use the program in batch mode. It also illustrates
a feedback mechanism, that not only allows the treatment of a large number of 
complicated partial differential equations but also aids in solving
the determining equations step by step.

\end{abstract}

\vfill
\newpage

\section*{PROGRAM SUMMARY}
\begin{tabular}{ll}

{\em Title of the program:} & SYMMGRP.MAX \\
{\em Catalogue Number:} & - - - \\
{\em Program obtainable from:} & CPC Program Library, \\
  & Queen's University of Belfast, \\
  & N. Ireland \\
  & (see application form in this issue) \\
{\em Computer \& Installation:} & VAX 11/750;  \\
  & Centre de Calcul, \\
  & Universit\'{e} de Montr\'{e}al, Montr\'{e}al, \\
  & Qu\'{e}bec, Canada H3C 3J7 \\
  & VAX 11/780; \\
  & Center for the Mathematical Sciences, \\
  & University of Wisconsin-Madison, \\
  & Madison, WI 53705, USA \\
  & VAX 8600; \\
  & Computer Center, Colorado School \\
  & of Mines, Golden, CO 80401, USA \\ 
{\em Operating system:} & VMS 3.7 or higher \\
{\em Programming language used:} & REX MACSYMA 305, \\
  & compatible with MACSYMA 309 and 412 \\
{\em High speed storage required:} & Problem dependent, \\
  & typical working set size about 0.5 M bytes \\
{\em Number of bits in a word:} & 32 \\
{\em Number of lines in program:} & 688  \\
{\em Card punching code:} & ASCII \\
{\em Keywords:} & Symmetry group, differential equations, \\
  & Lie algebras, symbolic computation,\\
  & MACSYMA \\
{\em Nature of the physical problem:} &    \\
\end{tabular}
\nopagebreak
\vskip 1pt
\noindent
The symmetry group of a given system of differential equations modeling
a physical phenomenon may be used to achieve several goals. 
These include the classification of the solutions of the system,
the generation of new solutions from known ones, 
the simplification of the system
by the method of symmetry reduction, to name a few. In the case where
particular symmetries must be present, the symmetry group can be used to
determine the validity of the modeling differential equations.

\pagebreak
\noindent
{\em \ Method of solution:}
\vskip 2pt
\noindent
The construction of the symmetry groups of
differential equations is based on an adaptation of the method, the
terminology and the method described in \cite{olverbook}. 
This procedure is translated into a MACSYMA 
\cite{macsymamanual,macsymausersguide} 
program that performs the most elaborate part of the job, namely
the construction of a complete list of determining equations which
is free of redundant factors, repetitions and trivial differential 
consequences.
\vskip 2pt
\noindent
{\em \ Restrictions on the complexity of the problem:}
\vskip 2pt
\noindent
For complicated systems of differential equations involving
derivatives of high order, time limits and available computer memory
may cause restrictions. Further limitations are discussed in Section 3.5 of
the Long Write-Up.
\vskip 2pt
\noindent
{\em \ Unusual features of the program:}
\vskip 2pt
\noindent
The flexibility of this program and the
possibility of using it in a partly interactive mode, allow one to find the 
symmetry groups of essentially arbitrarily large systems of equations.  
This is the main justification for
presenting a new symbolic manipulation program in a field where several
programs already exist.  Furthermore, this program has
been in use (at the Universit\'{e} de Montr\'{e}al and elsewhere) for over five
years, it has been tested on hundreds of systems of equations and has 
thus been comprehensively debugged.
\vskip 2pt
\noindent
{\em \ Running time:}
\vskip 2pt
\noindent
Given a system of $m$ differential equations of order $k$ with $q$ unknowns
and $p$ independent variables, running time is an increasing functions of
$m, k, q$ and $p$. Typical running times (CPU) for an example are given 
in Section 4.

\vfill
\newpage
\pagenumbering{arabic}

\section*{LONG WRITE-UP}
\section{Introduction}
\subsection{The problem}

For the purpose of this article the 'symmetry group of a system of differential
equations' is the largest local Lie group of local point transformations,
acting on the independent and dependent variables of the equation and leaving
the solution set of the system invariant. The symmetry group thus transforms
solutions of the system amongst each other.
A large body of old and new literature exists on this topic; here we just
refer to some recent books and reviews 
\cite{olverbook,amesbook}-\cite{winternitzkluwer}
We also recommend the special issues of {\em Acta Applicandae Mathematicae}
on 'Symmetries of Partial Differential Equations' \cite{specialissues}. 
\vskip 2pt
\noindent
A major obstacle in the application of Lie group theory to solving differential
equations is that usually a large number of tedious calculations is involved.
The purpose of this article is to present and make available a MACSYMA program
for the computer assisted calculation of symmetry groups.

The setting is an entirely general one and the method is well known and
described e.g. in \cite{olverbook}. 
\vskip 2pt

\nopagebreak
We consider a system of $m$ differential equations 
\begin{equation}
{\Delta}^{i} (x, u^{(k)})= 0, \;\;\;\; i = 1,2, ... , m, \label{orgsystem}
\end{equation}
of order $k$, with $p$ independent and $q$ dependent real
variables, denoted by
\begin{eqnarray}
x & = & (x_1 , x_2, ... , x_p ) \in {\R}^p , \label{variablesx}\\
u & = & (u^1 , u^2 , ... , u^q ) \in {\R}^q. \label{variablesu}
\end{eqnarray}

\nopagebreak
We stress that $m, k, p $ and $q$ are {\em arbitrary} positive integers.
The group transformations have the form
\begin{equation}
\tilde{x} = \Lambda_g (x,u), \;\;\;\; \tilde{u} = \Omega_g (x,u), 
\label{transform}
\end{equation}
where the functions 
$ \Lambda_g$ and $ \Omega_g $ are to be determined. Note that the subscript
$g$ refers to the group parameters. 
The approach is an infinitesimal one; instead of looking for a Lie group
$G$, we look for its Lie algebra $\cal L$, realized by vector fields of the
form 
\begin{equation}
\alpha = \sum_{i=1}^{p} \eta^i (x,u) 
\frac{\partial}{\partial x_i} + \sum_{l=1}^{q} \varphi_{l} (x,u)
\frac{\partial}{\partial u^{l}}.
\label{vectorfield}
\end{equation}
\pagebreak
The procedure \cite{olverbook} for finding the coefficients
$\eta^i (x,u)$ and $\varphi_l (x,u)$ 
is described below.
In essence, the computer constructs the $k^{\rm th}$ prolongation 
${\rm pr}^{(k)} \alpha $ of the vector field $\alpha$, applies it to the
system of equations (\ref{orgsystem}) and requests that the resulting 
expression vanishes on the solution set of (\ref{orgsystem}),
\begin{equation}
{\rm pr}^{(k)} \alpha \Delta^i  \mid_{\Delta^{j} = 0} 
\;\;\;\; i,j  = 1, ... , m. 
\label{prolongation}
\end{equation}

\nopagebreak
The result of implementing (\ref{prolongation}) is a system
of linear homogeneous PDEs
for $\eta ^i $ and $\varphi _l ,$ in which $x$ and $u$ are independent
variables. 
These are the so-called {\em determining equations} for the symmetries of the
system.

\nopagebreak
The procedure thus consists of two major steps:
{\em deriving} the determining equations and {\em solving} them.

\subsection{Review of symbolic programs}
\nopagebreak
Several computer packages
\cite{schwarzsiam,schwarzprogram1}-\cite{champagneprogram}
exist for this purpose, and some other programs were written for specific
examples \cite{rosencransprogram}.

\nopagebreak
The well-documented REDUCE program developed by F. Schwarz
\cite{schwarzsiam,schwarzprogram1}-\cite{schwarzprogram4},
is definitely going the furthest in 
solving the determining equations with minimal intervention by the user.
This program, called SPDE, is distributed with version 3.3 of REDUCE
for various types of computers, ranging from PCs to CRAYs.
Schwarz also rewrote SPDE 
\cite{schwarzsiam,schwarzprogram4} for use with SCRATCHPAD II, a 
symbolic manipulation program developed by IBM.

\nopagebreak
Based on Cartan's exterior calculus, Edelen \cite{edelenprogram} and
Gragert and Kersten \cite{gragertetalprogram} did some pioneering work
in using REDUCE to calculate the classical Lie symmetries of differential
equations.
Kersten \cite{kerstenprogram1,kerstenprogram2} later developed a 
REDUCE software package for the
calculation of the Lie algebra of infinitesimal symmetries (and corresponding
Lie-B\"{a}cklund transformations) of an exterior differential system.
Eliseev {\em et al} \cite{eliseevprogram} wrote a REDUCE program
to generate (but not solve) the system of determining equations for 
point and contact symmetries.
Fedorova and Kornyak \cite{fedorovakornyakprogram1} generalized the 
algorithm to include the case of Lie-B\"{a}cklund symmetries.

\nopagebreak
Apart from packages in REDUCE, we should mention the FORMAC programs by
Fedorova and Kornyak \cite{fedorovakornyakprogram2} and Fushchich and
Kornyak \cite{fushchichkornyakprogram} that create the system of 
determining equations for the Lie-B\"{a}cklund symmetries and solves these
equations as far as possible. The FORMAC package CRACKSTAR developed by
Wolf \cite{wolfprogram} also allows investigation of Lie symmetries of
PDEs, besides dealing with dynamical symmetries of ODEs and the like.

\nopagebreak
The program LIE by Head \cite{headprogram} is based on version 4.12
of muMATH, running on IBM compatible PCs.
Head's program calculates and solves the determining equations
automatically. Interventions by the user are sometimes needed and
therefore are made possible.

\pagebreak
The SYMCON package written by Vafeades \cite{vafeadesprogram1} also uses
muMATH to calculate the determining equations (without solving them).
Furthermore, the program verifies whether the symmetry group is of variational
or divergence type and computes the conservation laws 
associated with the symmetries.

\nopagebreak
Unfortunately, these programs are confined to the
256$\;$K memory accessible by muMATH and can therefore presently not handle 
very large systems of equations. 
This limitation motivated Vafeades to rewrite his SYMCON program 
in MACSYMA syntax \cite{vafeadesprogram2}. Although this program is
similar in mission to ours, Vafeades' program requires quite a bit more
interaction by the user. 
Geoff Prince from LaTrobe University 
(Melbourne, Australia) is working on a 'translation' of
the source code of LIE into REDUCE.

\nopagebreak
The calculation of the Lie group by computer was also proposed
by Popov, who used the program SOPHUS for the calculation of conservation
laws of evolution equations
\cite{popovprogram}.

\nopagebreak
The package DELiA by Bocharov \cite{bocharovprogram} also runs 
on PC and claims to perform various tasks based on Lie's approach, such
as the computation of point symmetries, conserved currents and conservation
laws; simplification and partial integration of overdetermined systems
of differential equations, etc.
The marketing material that comes with the demonstration disk for DELiA 
does not specify any underlying symbolic manipulation package. 
We believe that the program is written in PASCAL.
In \cite{bocharovbronsteinprogram} Bocharov and Bronstein present 
SCoLAr, a package based on standard PASCAL, 
for finding infinitesimal symmetries 
and conservation laws of arbitrary systems of differential equations. 

\nopagebreak
To the best of our knowledge, no package is available yet for 
the calculation of Lie symmetries with MAPLE and MATHEMATICA.

\nopagebreak
For completeness, we mention the pioneering work by C. Wulfman and his 
master students Davison and Nagao \cite{wulfmanprogram1,wulfmanprogram2}. 
Already in the early seventies, 
Davison \cite{wulfmanprogram1} developed computer algorithms 
in SNOBOL, a now obsolete computer language, 
that could handle symbolic manipulations with differential operators. 
In 1980, Nagao \cite{wulfmanprogram2} wrote the computer program 
DETERMININGEQS (in Pascal) that could approximate Lie generators for 
dynamical systems. 

\nopagebreak
Last but not least, we discuss the programs written in MACSYMA, the
symbolic package our symmetry program is based upon.
Just as REDUCE, MACSYMA is currently available for various types of computers, 
ranging from PCs to various work stations and main frame computers 
(such as VAX) and it is used all over the world.

\nopagebreak
Apart from an earlier version of our program \cite{champagneprogram} and the 
work done by Rosencrans \cite{rosencransprogram},
their are only three other MACSYMA-based symmetry programs.
\pagebreak
\noindent
The MACSYMA version of SYMCON by Vafeades \cite{vafeadesprogram2} 
was discussed above. Schwarzmeier and Rosenau 
\cite{rosenauschwarzmeierprogram,schwarzmeierrosenauprogram} made a
program that calculates the determining equations in their simplest form, 
but does not solve them automatically. 

\nopagebreak
The program SYM\_DE by Steinberg \cite{steinbergprogram1,steinbergprogram2} 
was recently added to the out-of-core library of MACSYMA. The program solves
some (or all) of the determining equations automatically and if needed the
user can (interactively) add extra information. Currently, Steinberg is working
at the extension of his program so that it would include the calculation 
of generalized (i.e. derivative dependent) symmetries.

\subsection{The program symmgrp.max}

\nopagebreak
The present program, called SYMMGRP.MAX is a modification of a 
package \cite{champagneprogram} that has been extensively used over the
last five years at the University of Montr\'{e}al and elsewhere. 
It has been tested on hundreds of systems of equations and has thus been 
solidly debugged.
The flexibility of this program and the possibility of using it in a
partly interactive mode, allow to find the symmetry group of in principal
arbitrarily large and complicated systems of equations 
on relatively small computers.
These are the main justifications for presenting yet another 
new symbolic program in a field where several programs already exist.

\nopagebreak
The amount of interaction by the user will depend on the
complexity of the system of differential equations and on the capacity
of the computer used.
Our experience is that for systems of equations the most time consuming
part of the calculation (when done by hand) is the derivation of the
determining equations and the elimination of redundant equations from the
system. 

\nopagebreak
The actual solving of the determining equations can usually be done
by inspection, using elementary results from the theory of linear PDEs.
Solving them on a computer may be time consuming, since the simplest 
approach varies greatly from case to case. Furthermore, a computer
program may accidentally not catch the most general result 
and therefore may return an incomplete symmetry group. 
The authors are very aware of this problem which occurred in testing
some of the other existing programs ! 
Fortunately, as soon as the new programs by Schwarz \cite{schwarzsizeprogram} 
and Reid \cite{reidprogram1}-{\cite{reidprogram3}, both for the 
determination of the size of a symmetry group, become available this problem
will be easily detectable.

\nopagebreak
Let us briefly digress on this topic.
Indeed, Schwarz and Reid independently developed
algorithms to determine the size of a symmetry group of Lie (point) symmetries.
Schwarz's program in REDUCE \cite{schwarzsizeprogram} calculates the number of 
parameters if the group is finite and the number of unspecified functions 
and its arguments if the group is infinite. 

\pagebreak
Recently, Reid \cite{reidprogram1}-\cite{reidprogram3} took up 
the same task. His program SYMCAL \cite{reidprogram2},
written originally in MACSYMA and currently being converted into MAPLE, 
computes the dimension and the structure constants of the Lie 
symmetry algebra of any system of PDEs.
An extension of the algorithm \cite{reidprogram3} also allows to classify
differential equations (with variable coefficients) according to the
structure of their symmetry groups. Furthermore, the approach 
advocated by Reid applies to the determination of symmetries of Lie, contact, 
and Lie-B\"acklund type as well as potential symmetries.

\nopagebreak
In the interest of versatility and simplicity, our present MACSYMA program 
concentrates on deriving the determining equations. It does not solve them 
neither does it calculate the size of the symmetry group.
Nevertheless, we believe that our program has some distinguished features 
and advantages:
\nopagebreak
\vskip 1pt
\noindent
1)$\;$The equations (\ref{prolongation}) can be treated simultaneously,
i.e. ${\rm pr}^{(k)} \alpha $ can be applied to all $m$ equations 
in the system.
The output is then a system of determining equations that is partly solved.
This means that the program takes all 'first order equations with one term'
and their differential consequences and uses them to
simplify the remaining determining system. This greatly decreases the
number of equations to be solved manually.
\nopagebreak
\vskip 1pt
\noindent
2)$\;$If the computer can be expected to run out of space when applying the
prolongation to the system (\ref{orgsystem}), it is possible to apply
${\rm pr}^{(k)} \alpha $ to a subset of equations, for instance just to
one equation, say $ \Delta^{1} = 0 .$ When implementing the requirement
$\displaystyle {\rm pr}^{(k)} \alpha \Delta^1 \mid_{\Delta^j = 0} = 0 ,
(j=1,...,m)$  of course, the program takes into account the entire 
system (\ref{orgsystem}), not just the equation  $ \Delta^{1} = 0 $ itself. 
\nopagebreak
\vskip 1pt
\noindent
3)$\;$If the individual equations in the system (\ref{orgsystem})
are so complex that the computer still runs out of space, it is possible
to derive only a subset of the determining equations, e.g. those that
occur as coefficients of the highest derivatives in (\ref{prolongation}).
These are usually single term equations.
\nopagebreak
\vskip 1pt
\noindent
4)$\;$A feedback mechanism has been incorporated. Once some of the 
determining equations have been solved the information obtained about the 
coefficients $\eta^i $ and $ \varphi_l$ can be submitted to the computer, 
which will present a new and simplified system of determining equations. 
The new information 
usually includes that the $\eta {\rm 's} $ and $ \varphi {\rm 's}$ are 
independent of some variables or depend linearly on some of the 
other variables. This greatly simplifies further calculations and, 
after several runs, makes it possible to apply 
${\rm pr}^{(k)} \alpha $ to the entire system 
(\ref{orgsystem}), even for very complicated ones. The feedback 
mechanism can be used all the way to the end.
At the last stage, when the completely determined forms of the
$\eta {\rm 's} $ and $ \varphi {\rm 's}$ are submitted,
the program will print out that the number of determining equations is zero,
i.e. the solution is verified. Hence, the program also allows to verify
any solution previously calculated by hand or by means of other programs.

\pagebreak
To summarize, whenever $ {\rm pr}^{(k)} \alpha $  
can be applied sucessfully to the system (\ref{orgsystem}), 
or a subset thereof, it produces a complete list of
determining equations.  
This list is free of trivial factors, duplication and differential 
redundancies. 
If, however, due to memory, time or space limitations, a complete list of
determining equations cannot be obtained, it is still possible to
derive a subset of the determining equations.  In this case,
heuristics are used to extract relevant information from that
subset.  This information is then fed back
into the program which resumes its calculations.
\nopagebreak
\vskip 1pt
In this respect, 
the philosophy of the approach implemented in the present program is to 
follow the path that would be taken in a manual calculation. That is, 
obtain in as simple a manner as possible all one term equations, solve
them and feed the information back to the computer. This can be done by first
treating just one equation and usually by extracting only the coefficients
of the highest derivatives. All the necessary substitutions and simplifations 
leading to the new determining system, 
which are error-prone if done by hand, are carried out automatically 
and correctly by this program.

\nopagebreak
\section{Procedure for Computing the Determining Equations}
We closely follow the notations, the terminology and the method for symmetry 
analysis used in \cite{olverbook}
which are well adapted to computer programming. 
Recall that the independent and 
dependent variables for the system (\ref{orgsystem}) are denoted by 
(\ref{variablesx}) and (\ref{variablesu}), respectively. 
The partial derivatives of $u^l$ are represented using a
multi index notation: for $J = (j_1 , j_2 , ... , j_p ) \in {\N}^p $
we put
\begin{equation}
u_{J}^{l} \equiv \frac{\partial^{|J|} u^l}
{\partial {x_1}^{j_1}   \partial {x_2}^{j_2} ...  \partial {x_p}^{j_p}} \; ,
\label{ujl}
\end{equation}
where $|J| = j_1 + j_2 + ... + j_p$.
Finally, $u^{(k)}$ will denote a vector whose components are all the partial
derivatives of order $0$ up to $k$ of all the $u^l$.

\nopagebreak
Using these notations the procedure for obtaining the determining equations
involves the following five steps:
\nopagebreak
\vskip 1pt
\noindent
(1)$\;$Construct the $k^{th}$ prolongation of the vector field $\alpha$ 
in (\ref{vectorfield}) by means of the formula
\begin{equation}
{\rm pr}^{(k)} \alpha = \alpha 
+ \sum_{l=1}^{q} \sum_{J} \psi_{l}^{J} (x, u^{(k)})
\frac{\partial }{\partial u^{l}_{J} }, \;\;\;\; 1 \leq |J| \leq k,
\label{prolongationonalpha}
\end{equation}
where the coefficients $ \psi^{J}_{l} $ are defined as follows. 

\pagebreak
\noindent
The coefficients of the first prolongation are
\begin{equation}
\psi_{l}^{J_{i}} = D_{i} \varphi_{l} (x,u) - \sum_{j=1}^{p} 
u_{J_{j}}^l D_{i} \eta^{j} (x,u), \label{psijil}
\end{equation}
where $J_{i}$ is a $p-$tuple with $1$ on the $i^{\rm th}$ position 
and zeros elsewhere, and $D_{i}$ is the total derivative operator
\begin{equation}
D_i = \frac{\partial}{\partial x_i} 
+ \sum_{l=1}^{q} \sum_J u_{J + J_{i}}^{l} \frac{\partial}
{\partial u_{J}^{l}}, \;\;\;\; 0 \leq  |J| \leq k.  \label{totalderivative}
\end{equation}
The higher order prolongations are defined recursively as
\begin{equation}
\psi_{l}^{J+J_{i}} = D_{i} \psi^{J}_{l} - \sum_{j=1}^{p} u_{J+J_{j}}^{l}
D_{i} \eta ^{j} (x,u), \;\;\;\; |J| \geq 1.
\label{psijjil}
\end{equation}
\nopagebreak
\vskip 1pt
\noindent
(2)$\;$Apply the prolonged operator ${\rm pr}^{(k)} \alpha $ to each equation
$\Delta^i (x, u^{(k)} ) $ and require that
\begin{equation}
{\rm pr}^{(k)} \alpha \; \Delta^i \mid _{\Delta^{j} = 0} \;\;  = 0 \;\;\;\; 
i,j = 1, ... , m.
\label{prolongationonsystem}
\end{equation}
The meaning of condition (\ref{prolongationonsystem}) is 
that ${\rm pr}^{(k)} \alpha $ vanishes on the solution set of the system 
(\ref{orgsystem}).
Precisely, this condition assures that $\alpha$ is an infinitesimal
symmetry generator of the transformation (\ref{transform}), i.e. that 
$u(x)$ is a solution of (\ref{orgsystem})  whenever $\tilde{u} (\tilde{x} )$
is one.
\nopagebreak
\vskip 2pt
\noindent
(3)$\;$Choose $m$ components of the vector $u^{(k)}$, 
say $ v^1 , ... , v^m $,
such that:
\nopagebreak
\vskip 2pt
\noindent
(a) Each $v^i$ is equal to a derivative of a $u^l \;\; (l = 1, ..., q)$ 
with respect to at least one variable $x_i \;\; (i = 1, ... , p). $ \par
\noindent
(b) None of the $v^i$ is the derivative of another one in the set. \par
\noindent
(c) The system (\ref{orgsystem})  can be solved algebraically for the $v^i$ 
in terms of the remaining components of $u^{(k)}$, 
which we denoted by $w$.
Hence, 
\begin{equation}
v^i = S^i (x,w), \;\;\;\; i = 1, ... , m. \label{vi}
\end{equation}
\noindent
(d) The derivatives of $v^i$, \par
\begin{equation} 
v^{i}_{J} = D_{J} S^{i} (x,w), \label{vij}
\end{equation}
\noindent
where $D_{J} \equiv D_{1}^{{j}_{1}} D_{2}^{{j}_{2}} ... D_{p}^{{j}_{p}}$,
can all be expressed in terms of the components of $w$ and their derivatives,
without ever reintroducing the $v^i$ or their derivatives. \par

\nopagebreak
While the above procedure sounds complicated,
for all specific systems that 
have been considered the choice of the appropriate $v^i$ has been quite simple.

\pagebreak
\noindent
For instance, for a system of evolution equations
\begin{equation}
u^{i}_{t} (x_1 , ... , x_{p-1} , t ) = F^{i} (x_1 , ... , x_{p-1}, t,u^{(k)}),
\;\;\;\; i = 1, ... , m,
\label{evoleqs}
\end{equation}
where $u^{(k)}$ involves derivatives with respect to the variables $x_i$ 
but not $t$, the appropriate choice is clearly $ v^i = u_{t}^{i} .$
\nopagebreak
\vskip 2pt
\noindent
(4) Use (\ref{vi}) 
to eliminate all $v^i$ and their derivatives from the expression
(\ref{prolongationonsystem}), so that all the remaining variables 
are now independent of each other.
\nopagebreak
\vskip 2pt
\noindent
(5) $\;$ Obtain the determining equations 
for $ \eta^{i} (x , u ) $ and $ \varphi_l (x , u) $ 
by equating to zero the coefficients
of all functionally independent expressions in the remaining derivatives
$u_{J}^{l}$.
\vskip 2pt
The described procedure is well defined as long as the variables $v^i$ 
in (\ref{vi}) exist.
Furthermore, the length and complexity of the calculations increase rapidly
as $p, q, m $ and especially $k$ increase.

\nopagebreak
\section{Description of the Program}
We now present the MACSYMA program SYMMGRP.MAX that realizes the procedure in 
Section 2 and that
provides a set of determining equations for the Lie symmetries
of an arbitrary system of differential equations. 

\nopagebreak
For this program we used MACSYMA release 412.61, which is usually implemented 
on VAX computers operating under VMS.
Note however that the program contains nothing beyond standard MACSYMA
statements and it is therefore compatible with earlier versions of 
MACSYMA, e.g. REX MACSYMA 305 and MACSYMA 309.6 (running under UNIX).
The user is supposed to have minimal experience with MACSYMA.
Information about the syntax of MACSYMA and many examples of its use
may be found in \cite{macsymamanual,macsymausersguide}. 

\nopagebreak
The program SYMMGRP.MAX for which the source code is given further,
consists essentially of function definitions. 
In fact, an appropriate function is defined for each major
task in the process. As we will see later, these
functions may be used individually provided that care is taken with respect
to their arguments. 

\nopagebreak
The primary function is called {\bf SYMMETRY} and it may be considered
as the main program. Once called, it reads the data, sets up the
environment for the calculation and then goes through the major steps
in the calculation by sequentially calling the other functions.

\nopagebreak
In addition to function definitions, there is a set of statements at the
top of the program that serves many purposes throughout the execution.
This set contains replacement {\em RULE} definitions, 
{\em MATCHDECLARE} statements
and {\em PATTERN MATCHING} definitions.

\pagebreak
\subsection{Description of the principal functions}
\vskip 1pt
{\bf PROVF(F) }: Applies the $k^{th}$ prolongation of $\alpha$ in
(\ref{vectorfield}) to a function $F(x, u^{(k)})$ 
and outputs the result. 
\vskip 2pt
\noindent
{\bf TOTDF(I,F)}: Applies the operator of total derivative $D_i$ defined in 
(\ref{totalderivative}) to a function $F(x, u^{(k)}) $ and
outputs the result.
\vskip 2pt
\noindent
{\bf FPSI(L,J)}: Calculates the coefficients $\psi^{J}_{l}$ in 
(\ref{prolongationonalpha})
through the use of the recursive formula (\ref{psijil}) and (\ref{psijjil}).
Actually, these coefficients are stored in an array PSI[L,J] defined by
PSI[L,J] := FPSI(L,J), so that one may call PSI instead of FPSI. 
The reason for this is that we can clear the array PSI without clearing 
the function 
definition used to calculate the $\psi^{J}_{l}$. Note that according to the
notation of Section 2, $J$ must be a MACSYMA list of $p$ integers.
\vskip 2pt
\noindent
{\bf EXTSUBST(EXPR)}: Applies to an expression EXPR depending on $x$ and 
$u^{(k)}$
the substitutions (\ref{vi}) and (\ref{vij})
until all the $v^i$ and their partial derivatives have been eliminated from
EXPR. It returns the new expression so obtained.
The function EXTSUBST may be used separately but the information concerning
the basic substitution (\ref{vi}) must be given. This is done by rewriting
(\ref{vi}) in the form
\begin{equation}
u^{l_{i}}_{{J}^{i}} = S^i (x,w), \;\;\;\; i = 1, ... , m,
\label{si}
\end{equation}
and according to this, defining the following arrays before using EXTSUBST:
\begin{equation}
\left.
\begin{array}{lll}
 INDJ[i] & : & J^i \\
 INDL[i] & : & l^i \\
 SUB[i]  & : & S^{i} (x,w) \\
\end{array}
\right\} i = 1, ... , m.
\label{indj}
\end{equation}
\nopagebreak
\noindent
{\bf SEARCHOEFF(EXPR)}: Given an expression EXPR which is a polynomial in the
derivatives of $u^l_J$ (exponents of the $u^l_J$ need not be integers but 
instead may be any real numbers), this function finds all the coefficients
of the various partial derivatives of $u$ and puts these coefficients in one
of two lists according to their length. More explicitly, at the end of the
procedure,
\nopagebreak
\vskip 2pt
\noindent
LODE[1]
will be a list containing all the coefficients which are 
monomials; 
\vskip 2pt
\noindent
LODE[2] will be a list containing all the coefficients which are 
polynomials (containing $'+'$ as the main operator). 
\nopagebreak
\vskip 2pt
Note that in the present context, these coefficients will be precisely the
determining equations.
\nopagebreak
\vskip 2pt
\noindent
{\bf SIMPEQN()}: Given the two lists of determining equations in $\eta^i$ and 
$\varphi_l$, LODE[1] and LODE[2], 
this function produces a unique list of
determining equations called LODE, equivalent to the union of the first two
lists but free of repetition and differential redundancy. More precisely,
\pagebreak
\begin{enumerate}
\item Equations of LODE will be ordered by increasing length relative to the
operator '+' (beginning with the monomials and ending with the longest
polynomials);
\item Monomial equations of LODE will be ordered by increasing order of 
differentiation with respect to $x_i$ and $u^l$;
\item The list of  equations LODE will be free of repetition and trivial
differential redundancy;
\item Any common factors, such as $x_i, u^l$, their products and powers 
thereof, occurring in the equations of LODE  will be factored 
out and canceled. 

\nopagebreak
The ELIMINATOR will also cancel all nonzero parameters 
(their products and powers)
given explicitly in the data file (see further). 
There will be a message for these
cancellations provided the parameter {\em warnings} is set to true. 
As a precaution, at the end of the simplifications the user is 
provided with a list of all the (non-trivial) factors that have been canceled 
during the execution of SIMPEQN. 
Special warnings messages are given if division
by parameters occurs. Note that the program will not clear sums or
differences of parameters, derivatives of functions, and the like.
This enables the user to determine how the symmetry group is affected by 
various choices of free parameters and functions.
\end{enumerate}
\nopagebreak
\phantom{junk junk junk junk}
\vskip 1pt
\indent
{\bf PRINTEQN(LIST)}: This function prints the elements of a LIST in
an equation like form (see 'Test Run Output' further on).
\nopagebreak
\vskip 2pt
\noindent
{\bf SYMMETRY(IND1,IND2,IND3)}: This function stands
for the main program. A call to it initiates the computation while
the three arguments enable the user to partially control the execution.
These arguments may take the values $0$ and $1$ and according to their values,
different actions are taken by the program as shown in Table 1.

\begin{center}
\begin{tabular}{||c|c|l||} \hline \hline
\multicolumn{3}{||c||}{\em Table 1 $\;\;\;$ Description of 
the arguments of SYMMETRY}  \\ \hline
Parameter &  Value & $\;\;\;\;\;\;\;\;\;\;\;\;\;\;\;\;\;\;\;\;$ 
Effect on the execution \\ \hline \hline 
IND1 & 0 & The program is used in interactive mode   \\ \hline
   & 1 & The program is used in batch mode   \\ \hline \hline
IND2 & 0 &The array PSI is cleared before the computation starts \\ \hline
   & 1 & PSI is not cleared, only new PSI will be computed \\ \hline \hline
IND3 & 0 & No trace of the calculations will be given\\ \hline
   & 1 & A trace of the calculations will be given  \\ \hline \hline
\end{tabular}
\end{center}

\pagebreak
\subsection{List of principal identifiers}
The correspondence between the identifiers used in the program 
and the mathematical symbols introduced in Sections 1 and 2 appear in Table 2.

\nopagebreak
\begin{center}
\begin{tabular}{||c|c|l||} \hline \hline
\multicolumn{3}{||c||}{\em Table 2 $\;\;\;$ 
List of the principal identifiers} \\ \hline
Text  & Program & $\;\;\;\;\;\;\;\;\;\;\;\;\;\;\;\;\;\;\;\;\;\;\;$ 
Meaning \\ \hline \hline 
$p $    &  $ P$  & Number of independent variables  \\ \hline
$q $    &  $ Q$  & Number of dependent variables  \\ \hline
$m $    &  $ M$  & Number of equations in the system  \\ \hline
$x_i $ & $ X[I]$ & Independent variables ($I \in \{ 1,2, ... , P \}$) \\ \hline
$u^l$  & $ U[L]$ & Dependent variables ($ L \in \{ 1,2, ... , Q \}$) \\ \hline
$J_{i}$ & $J[I] $   & Canonical vectors, e.g., $J[2]=[0,1,0,...,0] $  \\ \hline
$(j_1 , j_2, ... , j_p ) $ & $[J[1] , J[2], ..., J[p] ]$
&List of canonical vectors\\ \hline
$ u^{l}_{J}$ & $ U[L,J]$   & Derivatives of $u^{l}$  \\ \hline
$ \eta^{i}$ & ETA$[I]$    & Coefficient of 
                            $\textstyle \frac{\partial}{\partial x_i }$
in the vector field  \\ \hline
$ \varphi_{l}$ & PHI$[L] $ & Coefficient of 
                             $\textstyle \frac{\partial}{\partial u^l }$
in the vector field   \\ \hline
$ \psi^{J}_{l}$ & PSI$[L,J] $ & Derivatives of $\psi_{l}$  \\ \hline
$ \Delta^{i}$ &$EI$ & Equations in the system, $E1, E2,$ etc. \\ \hline
$ v^{i}$  & $VI$ & Variables for the substitution, $V1, V2,$ etc. \\ \hline
\hline
\end{tabular}
\end{center}

\nopagebreak
\subsection{Input data}
Every data file must have the following information:
\begin{enumerate}
\item The number of independent variables : $p$ (positive integer)
\item The number of dependent variables : $q$ (positive integer)
\item  The number of equations in the complete system : $m$ (positive integer)
\item The list of nonzero parameters (occurring in the given equations) that 
may be factored out and subsequently canceled: 
{\em parameters} : [a1,s1,aa,......,w2].
If there are no such parameters then {\em parameters} : [$\;$]. 
A discussion of names for parameters that may not be used is given
in Section 3.5.
\item The number of equations the program has to treat, controlled by
{\em sublisteqs}. 
For example, {\em sublisteqs}: [e1] for the first equation;
{\em sublisteqs}: [ei,...,em] for the 
$i^{th}$ through the $m^{th}$ equations only; 
{\em sublisteqs}: [all] for all the equations in the system

\pagebreak
\item The desired subset of the determining equations is controlled by
setting {\em highest\_derivatives}: $s$, where
$s$ stands for a positive integer indicating a count-down for the orders 
of derivatives in the prolongation. For example,
{\em 1} refers to information from the highest derivatives
only, {\em 2} for highest and second highest derivatives, etc., 
{\em all} will result in all the determining equations since all the 
terms in the prolongation were considered
\item The flag {\em warnings} : true or false,
controls the printout of messages about division by parameters 
and other simplifications 
\item The flag {\em info\_given }: false or true,
is used in connection with the feedback mechanism for solving the 
determining equations.
Information about the coefficients $\eta^i$ and the $\varphi_l$ 
in the vector field $\alpha$ must be entered in a specific way as 
discussed and exemplified in Section 4
\item The real equations $\Delta^i$ in (\ref{orgsystem})
must be given as $ei: ........ , $
with $ i = 1, ... , m $. You only have to put in the left hand side of the
equation, leaving out {\em '= 0'}. For complex equations see Section 3.5
\item The variables $v^i$ in (\ref{vi}), 
chosen for the substitution are given as $vi: ........, $
with $i = 1, ... , m. $
\end{enumerate}

\nopagebreak
Whether the program is used in batch mode or interactively, 
the data must be entered in an appropriate way:
\vskip 2pt
\noindent
{\bf (1) ~Batch mode}: 

\nopagebreak
This mode is definitely recommended for equations of 
fairly high order or for systems of equations (see Section 4).
A simple batch file contains the MACSYMA commands necessary 
to run the program and to read the data (most often given in a separate file). 
The data are put into a file with the use of MACSYMA assignment
statements.
This file is read before the function SYMMETRY is called (with IND1 set
to 1).

\nopagebreak
\noindent
{\bf (2) ~Interactive mode}:

\nopagebreak
This mode is invoked by calling SYMMETRY with IND1 set to zero.
It is only useful for single equations or fairly simple systems, e.g.
for equations of rather low order of derivation.
All parameters can be controlled interactively.
The program will prompt for the ten items listed above under 'Input data'.

\nopagebreak
If info\_given is set to true, the
program will prompt the user to put in the explicit forms of all the $\eta 's$
and the $\varphi 's.$ 
If for some of these functions no explicit information is available yet, 
then one simply submits their names, e.g. eta1, phi1, etc. 
The array notation is no longer allowed when info is given,
one has to use the 'concatenated' notation! 
Information about dependencies must be given with a DEPENDS statement
before calling the
function SYMMETRY, otherwise dependencies have to be stated explicitly as
arguments of the functions.

\pagebreak
\subsection{Output data}
At the end of the computation the determining equations are not 
automatically printed
but instead they are stored in a list of determining equations called LODE. 
A printout of the determining equations can be obtained with the command 
PRINTEQN(LODE).

\nopagebreak
For convenience, the function PRINTEQN is provided separately under the
name PRINTEQN.MAX (see further). 
This allows to reprint a previously saved list of 
determining equations (LODE).
If one wishes to use any other of the special functions described in 
Section 3.2 separately, then SYMMETRY must be called first with input
data consisting only of the $p, q$ and with $0$ assigned to $m$.

\nopagebreak
Table 3 summarizes the information available at the end of
the computation and how to access it. 

\nopagebreak
\begin{center}
\begin{tabular}{||l||l|l|l||} \hline \hline
\multicolumn{4}{||c||}{\em Table 3 $\;\;\;$ Information available at the end
of the computation}  \\ \hline
Identifier   &  Type & Printout & 
$\;\;\;\;\;\;\;\;\;\;\;\;\;\;\;\;\;\;\;\;\;\;\;$Contents \\ \hline \hline 
LODE & List & printeqn(lode);  
& List of the determining equations   \\ \hline
PSI  & Array & listarray(psi); 
& $\displaystyle \psi^{J}_{l}$ evaluated during the computation   \\ \hline 
ALGSOLS & List & printeqn(algsols); 
& Algebraic solution of the equations  \\ 
      &     &  & $\Delta^{i}=0$ for the variables $v^i$ \\ \hline
DIFSUB & Array & listarray(difsub); &
Substitutions used to eliminate the $v^i$ \\ 
      &        &  & and their derivatives  \\ \hline 
ETA & Array & listarray(eta); & Components of ETA \\ \hline
PHI & Array & listarray(phi); & Components of PHI \\ \hline \hline
\end{tabular}
\end{center}

\nopagebreak
\subsection{Type of systems}

\nopagebreak
As indicated previously, the program can be applied to systems of
$m$ differential equations of order $k$, with $p$ independent and $q$
dependent variables, where $m, k, p$ and $q$ are arbitrary positive
integers.  
Besides limitations due to allowable computer
memory and CPU time, the main restriction imposed on the original system
(\ref{orgsystem}) is related to the substitution procedure described 
in step (3) of Section 2.  
Indeed, it must be possible for MACSYMA to solve
the system algebraically for the $v^i$, as in (\ref{vi}).
For instance, this might prevent the application of the program
to a system involving five-fold nested radicals of the $u^{l}_{J}.$
Note, however, that in many applications, difficulties of this
type can be avoided by a judicious choice of the $v^{i}.$
It is important to emphasize that the individual
equations composing the system need not be polynomial in
the $x_i, u^l$ and $u^{l}_{J}.$  

\pagebreak
If only multiplicative and additive combinations of powers (not necessarily
integer ones) of the $u^{l}_{J}$ appear in (\ref{prolongationonsystem})
after the substitution procedure, then the determining equations can be used 
directly. 
If irrational or transcendental functions of the $u^{l}_{J}$ appear in 
(\ref{prolongationonsystem}), because they were present in the
original differential 
equations or they were introduced during the computations, then the
determining equations in the output will also contain some of the $u^{l}_{J}$
in a irrational or transcendental way. In that case, the user has the task
of obtaining the final list of independent determining equations from the
output.

\nopagebreak
The system of differential equations for which one wants to compute the
determining equations may contain arbitrary parameters and even arbitrary
functions of the variables $x_i$ and $u^l$.  However, names for these 
parameters or functions should not be in conflict
with any other identifiers used in the program: in particular, $I, J, L,
M, P, Q $ and $E1, E2, ... , V1, V2, ... $ may NOT be used. 
Note that MACSYMA is not sensitive to cases, e.g. v1 is the same as V1, etc.
For arbitrary functions, the dependencies must be declared with the help 
of a DEPENDS statement before calling SYMMETRY.

\nopagebreak
The program assumes that the $u^l$ are functions from ${\R}^p \mapsto \R$.
If a system of differential equations contains complex valued dependent
variables, i.e. functions from ${\R}^p \mapsto \C$, the decomposition 
into real and imaginary parts must be made before using the program 
(see Section 4).

In entering the original system of differential equations,
any use of C1, C2, ... , D1, D2, ... , must be avoided since it may 
confuse MACSYMA which uses these symbols to denote command and display lines, 
respectively. Furthermore, Greek letters should be avoided 
(for instance 'beta' refers to the Beta function and
'gamma' to the Gamma function). Before using any special character quickly 
check the index of \cite{macsymamanual,macsymausersguide} to verify
if the name or symbol does not interfere with a function or command name
in MACSYMA.

\nopagebreak
When the feedback mechanism is used, avoid any confusing 
between symbols and parameters occurring in the original equations and those 
used in the explicit forms for the $\eta 's$ and $\varphi 's.$

\nopagebreak
Note that MACSYMA starts
the label for an array at $0$. For instance, the first component of the array 
ETA is ETA[0]. Precautionary, we have assigned the 
value $0$ to the first components of such arrays. 

\pagebreak
\section{Example: The Karpman Equations}

\nopagebreak
This example shows how the program SYMMGRP.MAX can be used in batch mode. 
It also illustrates the feedback mechanism for solving 
the determining equations.

\nopagebreak
The Karpman equations \cite{karpman1} describe the effect of modulation 
instability of a high frequency (whistler) wave due to 
its resonant parametric interaction with a low frequency wave in a plasma.
The normalized complex amplitude $\psi$ of the whistler wave and the particle
density $\nu$ of the plasma are given by the nonlinear system:
\begin{eqnarray}
i ( {\psi}_t + w_1 \; {\psi}_z ) + \frac{1}{2}  
[ s_1 \; ( \psi_{xx} + \psi_{yy} ) + s_2\; \psi_{zz}] - a_11 \; \nu \; \psi 
& = & 0, \nonumber \\
\nu_{tt} - {(w_{2})}^2 \; ( \nu_{xx} + \nu_{yy} + \nu_{zz} ) 
- a_2 \; ( {\left| \psi \right|}^{2}_{xx} 
+ {\left| \psi \right|}^{2}_{yy} + {\left| \psi \right| }^{2}_{zz} ) &= & 0, 
\label{karpmaneqs}
\end{eqnarray}
where $ a_{1}, a_{2} , s_{1} , s_{2} , w_{1} $ and $w_{2}$ are just constants.

\nopagebreak
Since the program cannot handle the complex variable $\psi$, 
we split it in its real and imaginary parts by putting
\begin{equation}
\psi = \rho (x,y,z,t) \exp ( i \phi ( x,y,z,t) ) .
\label{realandimag}
\end{equation}

\nopagebreak
Thus, 
\begin{eqnarray}
\phantom{2} \!\!\!\!\!\!\!\!\!\! {\rho}_t \!+\! w_1 {\rho}_z 
\!+\! \frac{1}{2} \left[ s_1 ( 2 \rho_{x} \phi_{x} \! +\! 2 \rho_{y} \phi_{y} 
\! + \! \rho \phi_{xx} \! +\! \rho \phi_{yy} )
\! + \! s_2 ( 2 \rho_{z} \phi_{z} \! + \! \rho \phi_{zz} ) \right]\!\!\!\!\!
&=& \!\!\!\!\! 0,\phantom{22}
\nonumber \\
\phantom{2} \!\!\!\!\!\!\!\!\!\! {\phi}_t 
\! + \! w_11 {\phi}_z \! -\! \frac{1}{2}  
\left[ s_1 ( \frac{ \rho_{xx} }{ \rho } \! +\! \frac{ \rho_{yy} }{ \rho } 
\! - \! {\phi_{x}}^2 \! -\! {\phi_{y}}^2 )  
\! +\! s_2 ( \frac{ \rho_{zz} }{ \rho } \! -\! {\phi_{z}}^2 ) \right] 
\! +\! a_1 \nu \!\!\!\!\! &=& \!\!\!\!\! 0, \label{nkarpmaneqs} \\
\phantom{2} \!\!\!\!\!\!\!\!\!\! \nu_{tt} \! -\! {(w_{2})}^2 ( \nu_{xx} 
\! +\! \nu_{yy} \! +\! \nu_{zz} ) 
\! -\! 2 a_2 \rho ( \rho_{xx} \! +\! \rho_{yy} \! +\! \rho_{zz} )
\! -\! 2 a_2 ( {\rho_{x}}^2 \! +\!  {\rho_{y}}^2 
\! +\! {\rho_{z}}^2 ) \!\!\!\!\! &=& \!\!\!\!\! 0.\phantom{22}
\nonumber
\end{eqnarray}

\nopagebreak
For the system (\ref{nkarpmaneqs})
the MACSYMA calculations are fairly lengthy and involved. On some
computers, in particular on PCs, it may not possible to run all the equations 
at once.

\nopagebreak
Even on main frame computers it takes a long time.
For example, on a VAX 8600, 
the determining equations in simplified form were obtained in about 4 hours
of CPU time.
On a VAX 8650 with a central memory of 96 Megabyte, this same calculation took
3 hours of CPU time.
The number of determining equations before simplification was
2321. After automatic simplifications only 69 determining equations were left.
The peak working size being limited to about 16400 pages,
MACSYMA 412.61 needed 100 garbage collections due to 
dynamic 0 and 1 space overflow.

\pagebreak
For users of less sophisticated computers or when working with still larger
systems of equations, the strategy is to break up the problem in 
smaller pieces. 
The idea is to obtain information about independencies as soon as possible 
and to submit that information with a subsequent run.
This is done with a judicial setting of the parameters and with the
feedback mechanism. We illustrate this in all details for the
system (\ref{nkarpmaneqs}).

\nopagebreak
For this example the number of independent variables is $p=4$, 
the number of dependent variables is $q = 3$ and there are clearly $m=3$ 
equations.
The correspondences are as follows:
\begin{equation}
\begin{array}{rclcrcl}
x & \longmapsto & x[1], & \quad \quad & \rho  & \longmapsto & u[1] \; ,\\ 
y & \longmapsto & x[2], & \quad \quad &  \phi & \longmapsto & u[2] \; ,\\
z & \longmapsto & x[3], & \quad \quad &  \nu &  \longmapsto & u[3] \; , \\ 
t & \longmapsto & x[4]. & \quad \quad  & \quad & \quad & \quad  \\ 
\end{array}
\end{equation}
This permits to write the equations (\ref{nkarpmaneqs}) 
in a standard form accepted by the program (see below under "e1" 
through "e3").
Finally, one selects the variables needed for the substitution 
(elimination) process:
these will be $\rho_t , \phi_t$ and $\nu_{tt} .$
In the notation of the program, these variables are called 
$v1, v2$ and $v3$.

\nopagebreak
The 'translation' of the equations (\ref{nkarpmaneqs}) 
is given in the data file {\it KARPMANRUN1.DAT} below. 
For example, $ \rho_{t}$ becomes $u[1,[0,0,0,1]],$ etc.

\nopagebreak
The batch file containing the MACSYMA commands to run 
SYMMGRP.MAX is called {\em KARPMANRUN1.COM}. It contains:
\vskip 1pt
batchload("symmgrp.max"); \par
writefile("karpmanrun1.412"); \par
batch("karpmanrun1.dat"); \par
symmetry(1,0,0); \par
printeqn(lode); \par
save("lodekarpmanrun1.lsp",lode); \par
closefile(); \par
quit(); \par

\nopagebreak
\noindent
For every new run this batch file has to be slightly updated by changing
'run1' into 'run2', etc.
As for its contents, apart from saving a transcript of the session in
{\em KARPMANRUN1.412}, we also save the list of determining equations (lode)
as the LISP file {\em LODEKARPMANRUN1.LSP}, 
in case one wants to use these equations in a separate MACSYMA session. 
In turn, the above command file batches the file {\em KARPMANRUN1.DAT} with the
data for the first run:
\nopagebreak
\vskip 2pt
p : 4 \$ \par
q : 3 \$ \par
m : 3 \$ \par
\pagebreak
parameters : [a1,a2,s1,s2,w1,w2] \$ \par
warnings : true \$ \par
sublisteqs : [e1] \$ \par
info\_given : false \$ \par
highest\_derivatives : 1 \$ \par
e1 : u[1,[0,0,0,1]]+w1*u[1,[0,0,1,0]]+(1/2)*(s1*(2*u[1,[1,0,0,0]] \par
$\;\;\;\;\;$
*u[2,[1,0,0,0]]+2*u[1,[0,1,0,0]]*u[2,[0,1,0,0]]+u[1]*u[2,[2,0,0,0]] \par
$\;\;\;\;\;$
+u[1]*u[2,[0,2,0,0]])+s2*(2*u[1,[0,0,1,0]]*u[2,[0,0,1,0]] \par
$\;\;\;\;\;$
+u[1]*u[2,[0,0,2,0]])); \par
e2 : u[2,[0,0,0,1]]+w1*u[2,[0,0,1,0]]-(1/2)*(s1*(u[1,[2,0,0,0]]/u[1] \par
$\;\;\;\;\;$
+u[1,[0,2,0,0]]/u[1]-u[2,[1,0,0,0]]\^{}2 -u[2,[0,1,0,0]]\^{}2 ) \par
$\;\;\;\;\;$
+s2*(u[1,[0,0,2,0]]/u[1]-u[2,[0,0,1,0]]\^{}2 ))+a1*u[3]; \par
e3 : u[3,[0,0,0,2]]- v2\^{}2 *(u[3,[2,0,0,0]]+u[3,[0,2,0,0]]
+u[3,[0,0,2,0]]) \par
$\;\;\;\;\;$
-2*a2*u[1]*(u[1,[2,0,0,0]]+u[1,[0,2,0,0]]+u[1,[0,0,2,0]]) \par
$\;\;\;\;\;$
-2*a2*(u[1,[1,0,0,0]]\^{}2 +u[1,[0,1,0,0]]\^{}2 
+u[1,[0,0,1,0]]\^{}2 ); \par
v1 : u[1,[0,0,0,1]]; \par
v2 : u[2,[0,0,0,1]]; \par
v3 : u[3,[0,0,0,2]]; \par
\vskip 2pt
\noindent
\nopagebreak
All the parameters a1, a2, .... w1, are supposed to be nonzero and 
may be canceled (as factors) in simplifications.
Since we selected only the first equation of the system (however substituting
from the entire system !) and since we extract only the simple determining 
equations (from the coefficients of the highest derivatives in the 
prolongation), this run takes only 20 minutes of CPU time on a VAX 8600.

\nopagebreak
The simplifications implemented in the program reduce the number of 
determining equations from 20 to 6 single term equations. They are saved in
KARPMANRUN1.412 and these equations reveal that
$eta4$ only depends on x[4] as listed in Table 4.
'No' stands for 'not dependent on' and the subscript {\em 1} refers to the 
information obtained from the {\em first} run, etc.

\nopagebreak
\begin{center}
\begin{tabular}{||c||c|c|c|c|c|c|c||} \hline \hline
\multicolumn{8}{||c||}{\em Table 4 $\;\;\;$ Dependencies for the Karpman Case} 
\\ \hline
     & x[1]   & x[2] &  x[3]   & x[4]    & u[1]   & u[2]   & u[3] \\ \hline 
\hline
eta1 &$no_4$ &       & $no_4$  & $no_4$  & $no_3$ & $no_3$ & $no_2$  \\ \hline
eta2 &       & $no_4$& $no_4$  & $no_4$  & $no_3$ & $no_3$ & $no_2$  \\ \hline
eta3 &$no_4$ & $no_4$& $no_4$  & $no_4$  & $no_3$ & $no_3$ & $no_2$  \\ \hline
eta4 &$no_1$ & $no_1$& $no_1$  & $no_4$  & $no_1$ & $no_1$ & $no_1$  \\ \hline
\hline
phi1 &      &        &         &         &        & $no_3$ & $no_2$  \\ \hline
phi2 &      &        &         &         &$no_3$  &${\rm linear}_3$&$no_3$
\\ \hline
phi3 &      &        &         &         &$no_4$  &$no_4$ & ${\rm linear}_4$
\\ \hline \hline
\end{tabular}
\end{center}
\vskip 2pt
\pagebreak
For the second run, we provide the program with this information and
we ask for the determining equations coming from the
second prolongation applied to e2 and e3 only.
We modify a few lines in the data file: 
\vskip 2pt
sublisteqs : [e2,e3] \$ \par
info\_given : true \$ \par
depends([eta1,eta2,eta3,phi1,phi2,phi3],[x[1],x[2],x[3],x[4],u[1],u[2],u[3]]);
\par
depends(eta4,x[4]); \par
\vskip 2pt
\nopagebreak
\noindent
All the remaining lines are the same and we save the updated file as
{\em KARPMANRUN2.DAT}. We run the program again with a batch
file similar to the one used for the first run.
After 1 h. 15 min. of CPU time, four determining equations are obtained, 
they give new information about the dependencies (see Table 4).
\vskip 2pt
For the third run, in {\em KARPMANRUN3.DAT} we
update the information about the dependencies 
(by changing dependency declarations as shown before) 
and we ask the program for all the determining equations coming from 
the first equation: 
\vskip 2pt
sublisteqs : [e1]\$ \par
highest\_derivatives : all\$ \par
\vskip 2pt
\noindent
In 6 min. of CPU time on the VAX 8600, the program extracted 130 determining 
equations and automatically simplified them to 26 equations.
The info from the first 7 (single term) equations is added to Table 4.
\nopagebreak
At this point we want to solve some of the 19 remaining equations, 
to prepare the data for the next run. 
\nopagebreak
The determining equations are all linear and homogeneous. So they 
usually do not require any solution techniques beyond a straightforward 
separation of variables, occasionally a simple integration,
at worst an application of the method of the characteristics.
\nopagebreak
We first look for more information on dependencies.
Since $phi1$ and $phi2$ are independent of $u[3]$ the (longest) equation, i.e.,
\begin{eqnarray}
&2&\!\!\! w1 \; \frac{\partial phi1}{\partial x[3]} 
+ s2 \; u[1] \; \frac{\partial^2 phi2}{\partial x[3]^2} 
+ s1 \; u[1] \; \frac{\partial^2 phi2}{\partial x[2]^2} \nonumber \\
&+&\!\!\! s1 \; u[1] \; \frac{\partial^2 phi2}{\partial x[1]^2} 
+ 2 \; \frac{\partial phi1}{\partial x[4]} 
- 2 a1 \; u[3] \; \frac{\partial phi1}{\partial u[2]} = 0, 
\label{eqdphi1du2}
\end{eqnarray}
implies that $phi1$ is also independent of $u[2]$.
With
\begin{equation}
u[1]^2 \frac{\partial phi2}{\partial u[1]} 
+ \frac{\partial phi1}{\partial u[2]} = 0,
\label{eqdphi2du1}
\end{equation}
we have that $phi2$ is independent of $u[1]$. 
\vskip 1pt
\pagebreak
\noindent
Next,
\begin{equation}
u[1] \frac{\partial^2 phi2}{\partial u[2]^2} 
+ \frac{\partial phi1}{\partial u[2]} = 0,
\label{phi2linear}
\end{equation}
gives that $phi2$ is linear in $u[2]$. We add these three conclusions 
to Table 4.

\nopagebreak
From the three remaining equations (with only two terms) we learn that
\begin{eqnarray}
\frac{\partial eta3}{\partial x[1]}& = & - 
\left( \frac{s2}{s1} \right) \frac{\partial eta1}{\partial x[3]},
\label{grad1} \\
\frac{\partial eta3}{\partial x[2]}& = & 
-\left( \frac{s2}{s1} \right) \frac{\partial eta2}{\partial x[3]},
\label{grad2} \\
\frac{\partial eta2}{\partial x[1]} & =& -\frac{\partial eta1}{\partial x[2]}. 
\label{grad3} \\ \nonumber
\end{eqnarray}

\nopagebreak
Comparison of three equations with 4 terms each, such as
\begin{equation}
\frac{\partial phi2}{\partial u[2]} 
+ u[1] \frac{\partial^2 phi2}{\partial u[1] \partial u[2]} 
+ \frac{\partial eta4}{\partial x[4]}
- 2 \frac{\partial eta3}{\partial x[3]} = 0,
\label{eq1phi2}
\end{equation}
leads to 
\begin{equation}
\frac{\partial eta3}{\partial x[3]}= \frac{\partial eta2}{\partial x[2]} = 
\frac{\partial eta1}{\partial x[1]}.
\label{grad4}
\end{equation}
Upon integration of (\ref{eq1phi2}) we get
\begin{equation}
phi2 = (2 \frac{\partial eta1}{\partial x[1]} 
- \frac{\partial eta4}{\partial x[4]} ) \; u[2]  
+ f2 (x[1], x[2], x[3], x[4]),
\label{phi2}
\end{equation}
where  $f2$ will be determined later.
\vskip 1pt
\noindent
Substitution of (\ref{phi2}) into
\begin{equation}
u[1] \frac{\partial phi2}{\partial u[2]} 
- u[1] \frac{\partial phi1}{\partial u[1]} + phi1 
+ u[1] \frac{\partial eta4}{\partial x[4]}
- 2 u[1]  \frac{\partial eta1}{\partial x[1]}
= 0
\label{eq1phi1}
\end{equation}
and integration yields
\begin{equation}
phi1 = f1(x[1], x[2], x[3], x[4]) \; u[1],
\label{phi1}
\end{equation}
where $f1$ will be determined later.
\vskip 1pt

\nopagebreak
To save time we shall not solve the rest of the equations for 
$eta1, eta2, eta3$ and $phi3$ 
but rather submit the above information and carry out the next run.

\nopagebreak
Hence, the data file {\em KARPMANRUN4.DAT} contains
the information from Table 4 and also the lines
\vskip 2pt
\pagebreak
sublisteqs:[all] \$ \par
depends([f1,f2],[x[1],x[2],x[3],x[4]]); \par
phi1 : f1*u[1]; \par
phi2 : (2*diff(eta1,x[1])-diff(eta4,x[4]))*u[2]+f2; \par
gradef(eta3,x[1],-(s2/s1)*diff(eta1,x[3])); \par
gradef(eta3,x[2],-(s2/s1)*diff(eta2,x[3])); \par
gradef(eta2,x[1],-diff(eta1,x[2])); \par
gradef(eta3,x[3],diff(eta1,x[1])); \par
gradef(eta2,x[2],diff(eta1,x[1])); \par
\vskip 2pt
\noindent
\nopagebreak
After 28 min. of CPU time, 30 simple determining equations are obtained
(see 'Test Run Output'). 
The simplifications described in the outline of the program 
actually reduced 249 determining equations to 30 this time.
\vskip 1pt
\noindent
Since $s1 \not= s2$, 10 of these equations together with the
conditions (\ref{grad1})-(\ref{grad3}) and (\ref{grad4}) 
lead to the information listed in Table 4.

\nopagebreak
A quick inspection of the remaining equations in 'Test Run Output'
allows to conclude that $f1 = k6 $ is constant.
Hence, with (\ref{phi1}) we get
\begin{equation}
phi1 = k6 \; u[1] \label{phi1new}.
\end{equation}

\nopagebreak
Similarly, (\ref{phi2}) simplifies into
\begin{equation}
phi2 = f2(x[4]) \label{phi2new}.
\end{equation}
Further, we obtain $phi3$ up to an unknown function 
$f4$,
\begin{equation}
phi3 = 2 \; k6\; u[3] + f4(x[1],x[2],x[3],x[4]).
\label{phi3}
\end{equation}
We also find that $eta1$ is linear in $x[2]$, i.e.,
\begin{equation}
eta1 = k1 \; x[2] + k2,
\label{eta1}
\end{equation}
where $k1$ and $k2$ are constants. 
The equations (\ref{grad1})-(\ref{grad3}) and (\ref{grad4}) 
then determine
\begin{eqnarray}
eta2 & =& - k1 \; x[1] + k3,
\label{eta2} \\
eta3 &=& k4,
\label{eta3} \\
eta4 &=& k5.
\label{eta4} \\ \nonumber
\end{eqnarray}
\nopagebreak
\noindent
We again modify the previous data file, to account for the info
in Table 4, the forms of the eta's and phi's and the dependencies of 
$f2$ and $f4$:
\vskip 2pt
\pagebreak
depends(f2,x[4]); \par
depends(f4,[x[1],x[2],x[3],x[4]]); \par
\par
eta1 : k1*x[2]+k2; \par
eta2 : -k1*x[1]+k3; \par
eta3 : k4; \par
eta4 : k5; \par
phi1 : k6*u[1]; \par
phi2 : f2; \par
phi3 : 2*k6*u[3]+f4; \par
\noindent
and with this file {\em KARPMANRUN5.DAT} we start the last run.
Only 2 determining equations are left 
in {\em KARPMANRUN5.412}:
\begin{equation}
2 \; u[3]\; a1 \; k6 + a1 \; f4
+ \frac{\partial f2}{\partial x[4]} = 0,
\label{k6}
\end{equation}
\begin{equation}
  {w2}^2 \frac{\partial^2 f4}{\partial x[3]^2} 
+ {w2}^2 \frac{\partial^2 f4}{\partial x[2]^2} 
+ {w2}^2 \frac{\partial^2 f4}{\partial x[1]^2} 
-\frac{\partial^2 f4}{\partial x[4]^2} = 0.
\label{f4}
\end{equation}
The first one requires that $k6=0$, hence, $phi1 = 0, $ and also
\begin{equation}
\frac{\partial f2}{\partial x[4]} = - a1 \; f4.
\label{f4new}
\end{equation}
Since $f2$ depends only on $x[4]$, $f4$ must be independent of 
$x[1], x[2]$ and $x[3]$. 
As a consequence of (\ref{f4}), $f4$ is linear in $x[4]$ and with
(\ref{f4new}) the final solution is known:
\begin{eqnarray}
f2 &=& a1 \; k7 \; {x[4]}^2 + a1 \; k8 \; x[4] + k9, \label{f2final} \\
f4 &=& - 2 \; k7 \; x[4] - k8, \label{f4final} \\ \nonumber
\end{eqnarray}
where $k7, k8$ and $k9$ are free constants. 
These functions determine the final form of $phi2$ and $phi3$ in
(\ref{phi2new}) and (\ref{phi3}).
\vskip 2pt
One could submit these data for verification. We have done so and no
determining equations were left, as expected. 
This final test of the solution took only 1 min. 30 sec. of CPU time.

\nopagebreak
Let us summarize. The general solution of the determining equations leads
to a Lie algebra with 8 infinitesimal generators. In terms of the original 
independent variables $x, y, z, t$ and the dependent variables $\rho, \phi $ 
and $\nu$, the vector field reads:
\begin{equation}
\alpha = \eta^{x} \frac{\partial}{\partial x}
+ \eta^{y} \frac{\partial}{\partial y}
+ \eta^{z} \frac{\partial}{\partial z}
+ \eta^{t} \frac{\partial}{\partial t}
+ \varphi^\rho \frac{\partial}{\partial \rho}
+ \varphi^\phi  \frac{\partial}{\partial \phi}
+ \varphi^\nu  \frac{\partial}{\partial \nu} ,
\label{coefvectorfield}
\end{equation}
\pagebreak
\noindent
where
\begin{equation}
\begin{array}{rclcrcl}
\eta^{x} & = & k_1 \; y + k_2 & \quad , \quad & \varphi^{\rho} & = & 0 \; , \\
\eta^{y} & = & - k_1 \; x + k_3 & \quad ,  \quad
& \varphi^{\phi} & = & k_7 \; {\rm a_1} \; t^2 +  k_8 \; {\rm a_1} \; t 
+ k_9 \; ,\\
\eta^{z} & = & k_4 &\quad ,\quad 
& \varphi^{\nu} & = & - 2 \; k_7 \;t - k_8 \; , \\
\eta^{t} & = & k_5 \; . &\quad \quad & \quad & \quad & \quad 
\end{array}
\end{equation}
\nopagebreak
\noindent
Here $k_1$ through $k_9$ are independent arbitrary constants ($k_6 = 0$
making $\varphi^{\rho} = 0 $).
Recall that  ${\rm a_1}$ is a parameter in the Karpman equations 
(\ref{karpmaneqs}).
The 8 infinitesimal generators for these equations are 
\samepage{
\begin{equation}
\begin{array}{rclcrcl}
P_1 & = & \partial_x & \quad , \quad & L_3 &=& y \partial_x - x \partial_y 
\; ,\\
P_2 & = & \partial_y &\quad ,\quad & R_1 & = & \partial_{\phi} \; , \\
P_3 & = & \partial_z & \quad ,\quad &R_2 & = & a_1\;  t \; \partial_{\phi} - 
\partial_{\nu} \; , \\
P_4 & = & \partial_t &\quad , \quad &R_3 & = & a_1\;  t^2 \; \partial_{\phi} 
- 2 t\; \partial_{\nu} \; .
\end{array}
\end{equation} }

\section{Conclusion}

We presented a MACSYMA program that can assist in the calculation of the
symmetry group of a system of differential equations. Among various features
of this program, let us emphasize a few.\par
\noindent
(1) The program is applicable to a system of $m$ equations of order $k$, with
$q$ unknowns and $p$ independent variables, where all these labels are
arbitrary positive integers. \par
\noindent
(2) The output is a system of determining equations that is free 
of repetition and partially solved in the sense that higher order equations 
which are differential consequences of lower order ones are automatically 
eliminated. \par
\noindent
(3) The parameters 'sublisteqs' and 'highest\_derivatives' allow
partial information to be extracted very rapidly. 
These parameters help prevent MACSYMA
running out of space (and/or time) when very large systems 
are submitted. \par
\noindent
(4) Warnings remind the user about
division by parameters that were listed as different from  zero. \par
\noindent
(5) The feedback mechanism allows the determining equations 
to be solved step
by step on the computer, hence avoiding human error in the algebraic 
simplifications. \par
\noindent
(6) The program can be used to test solutions of the determining equations 
and hence detect errors in the literature on the subject. \par
\noindent
(7) The program can be used interactively and in batch mode and the
amount of messages that are printed out can be adequately controlled. \par
\noindent
(8) The program needs very little data and is straightforward to use
provided the user has access to MACSYMA. \par
\noindent
Application of this program to determine the symmetry group of the
Karpman equations has been straightforward and has lead to a results.
The development of a MACSYMA program that solves the determining
equations in part is planned for the future. 
Upon modification of the algorithm, the
program can be extended to the computation of more general 
Lie-B\"{a}cklund transformation groups \cite{andersonibragimovbook}.

\vskip 2pt
\section*{Notes added in Proof}
\vskip 2pt
\begin{itemize}
\item One new parameter must be added to the input data as described in 
subsection 3.3 of this paper.
The parameter {\em subst\_deriv\_of\_vi}: true, controls the 
substitution of the partial derivatives of the $v^i$ in the equation (12). 
These derivatives are given by (14).

In some cases it is not possible to select the $v^i$ in such a way that
the differential consequences would not reintroduce lower order derivatives 
of the $v^i$, hence causing a loop!  Therefore, we have make the substitution 
of the partial derivatives of the $v^i$ optional. If only the $v^i$ should be 
replaced and not their derivatives, one puts 
{\em subst\_deriv\_of\_vi} : false.

The resulting system of determining equations is `equivalent' with the one
obtained using the substitution of all the partial derivatives of the $v^i$.
In the later case the system of determining equations may be somewhat simpler,
but the extra substitutions consume time.

\item The authors became aware of yet another REDUCE program for the 
calculation of Lie symmetries (including Lie-B\"{a}cklund symmetries) 
developed by Clara Maria Nucci at the School of Mathematics, Georgia Institute 
of Technology, Atlanta, Georgia. 
\end{itemize}

\vfill
\newpage
\section*{Acknowledgements}

This work was partially supported by research grants from the Natural
Science and Engineering Research Council of Canada and the "FCAR du
Gouvernement du Qu\'{e}bec". B.C. acknowledges a fellowship
awarded by NSERC of Canada. W.H. is grateful for support from
the Air Force Office of Scientific Research under Grant No. 85-NM-0263.
He also thanks the Centre de Recherches Math\'{e}matiques for their hospitality
and support.
The authors thank D. Rand for help with MACSYMA, and 
A. K. Head, D. Holm, M. Homer, A. Mikhailov, P. Olver, G. Prince, W. Sarlet and
S. Steinberg for bringing various software packages to their attention.
M. Grundland and D. Levi are gratefully acknowledged for using the
program extensively and suggesting improvements.

\vfill
\newpage

\begin{thebibliography}{99}

\bibitem{olverbook} {\sc P.J. Olver},\, 
{\em Applications of Lie Groups to Differential Equations}\, 
(Springer Verlag, New York, 1988).

\bibitem{macsymamanual} {\em MACSYMA Reference Manual}, Version 13,\,
Computer Aided Mathematics Group\, (Symbolics, Burlington, MA, 1989).

\bibitem{macsymausersguide} {\em MACSYMA User's Guide},\,
Computer Aided Mathematics Group\, (Symbolics, Burlington, MA, 1988).

\bibitem{amesbook} {\sc W.F. Ames},\, {\em  Nonlinear Partial Differential
Equations in Engineering}\, (Academic Press, New York, 1972).

\bibitem{blumancolebook} {\sc G.W. Bluman} and {\sc J.D. Cole},\, 
{\em Similarity Methods for Differential Equations} \,
(Springer, New York, 1974).

\bibitem{millerbook} {\sc W. Miller},\, {\em Symmetry and Separation of
Variables} \, (Reading, MA, 1977).

\bibitem{andersonibragimovbook} {\sc R.L. Anderson} and {\sc N.H. Ibragimov},\,
{\em Lie-B\"{a}cklund Transformations in Applications},\, Studies in 
Applied Mathematics {\bf 1} (SIAM, Philadelphia, 1979).

\bibitem{ovsiannikovbook} {\sc L.V. Ovsiannikov},\, {\em Group Analysis of 
Differential Equations} \, (Academic Press, New York, 1982).

\bibitem{winternitzlecnotes} {\sc P. Winternitz},\, 
in: K.B. Wolf,\, {\em Nonlinear Phenomena},\, Lecture Notes in Physics 189 \, 
(Springer Verlag, New York, 1983) p. 263.

\bibitem{ibragimovbook} {\sc N.H. Ibragimov},\, {\em Transformation
Groups Applied to Mathematical Physics} \, (Reidel, Boston, 1985).

\bibitem{kalninsbook} {\sc E.G. Kalnins},\, {\em Separation of Variables for
Riemannian Spaces of Constant Curvature}\, 
(Longman, Essex, 1986).

\bibitem{sattingerweaverbook} {\sc D.H. Sattinger} and {\sc O.L. Weaver},\,
{\em Lie Groups and Algebras with Applications to Physics, Geometry, and
Mechanics}\, (Springer Verlag, New York, 1986).

\bibitem{fushchichnikitin} {\sc V.I. Fushchich} and {\sc A.G. Nikitin},\, 
{\em Symmetries of Maxwell Equations}\,
(Reidel, Dordrecht, 1987).

\bibitem{leviwinternitzbook} {\sc D. Levi} and {\sc P. Winternitz},\, Eds.,\,
{\em  Symmetries and Nonlinear Phenomena}\,
(World Scientific, Singapore, 1988).

\newpage

\bibitem{schwarzsiam} {\sc F. Schwarz},\, 
SIAM Review 30 (1988) 450.

\bibitem{blumankumeibook}  
{\sc G.W. Bluman} and {\sc S. Kumei},\, 
{\em Symmetries and Differential Equations}\,
(Springer Verlag, New York, 1989).

\bibitem{rogersamesbook} {\sc C. Rogers} and {\sc W.F. Ames},\, 
{\em Nonlinear Boundary Value Problems in Science and Engineering}\,
(Academic Press, New York, 1989).

\bibitem{stephanibook} {\sc H. Stephani},\, 
{\em Differential Equations: Their Solution using Symmetries}\,
(Cambridge University Press, Cambridge, 1989).

\bibitem{winternitzkluwer} {\sc P. Winternitz},\, 
in: R. Conte and N. Boccara,\, {\em Partially Integrable Nonlinear 
Evolution Equations and their Physical Applications}\, 
(Kluwer, Dordrecht, 1990)\, p. 515

\bibitem{specialissues} {\sc A.M. Vinogradov},\, Ed.,\,
{\em Symmetries of Partial Differential Equations},\, 
Part I, Acta Appl. Math. 15, nos. 1 \& 2 (1989);\, 
Part II, {\em ibid.}\, 16, no. 1 (1989);\, 
Part III, {\em ibid.}, 16, no. 2 (1989).

\bibitem{schwarzprogram1} {\sc F. Schwarz},\,
Comput. Phys. Comm. 27 (1982) 179.

\bibitem{schwarzprogram2} {\sc F. Schwarz},\,
Computing 34 (1985) 91;\, Addendum: Computing 36 (1986) 279.

\bibitem{schwarzprogram3} {\sc F. Schwarz},\,
Comput. Phys. Comm. 39 (1986) 285.

\bibitem{schwarzmanual} 
{\sc F. Schwarz}
{\em The Package SPDE for Determining Symmetries of Partial Differential
Equations. User's Manual}.\, Distributed with REDUCE 3.3 
(Rand Corporation, Santa Monica, CA, 1987).

\bibitem{schwarzprogram4} {\sc F. Schwarz},\,
in: R. JanBen,\, {\em Trends in Computer Algebra},\,
Lecture Notes in Comput. Sci. 296 \,(Springer Verlag, New York, 1988)\, p. 167.

\bibitem{edelenprogram} {\sc D.G.B. Edelen},
{\em Isovector Methods for Equations of Balance} (Sijthoff \& Nordhoff,
Alphen aan de Rijn, 1981).

\bibitem{gragertetalprogram} {\sc P. Gragert, P.H.M. Kersten}
and {\sc A. Martini}, \,
Acta Appl. Math. 1 (1983) 43.

\bibitem{kerstenprogram1} {\sc P.H.M. Kersten},\,
{\em Infinitesimal Symmetries: a Computational Approach},\,
CWI Tract 34 (Center for Mathematics and Computer Science, Amsterdam, 1987).

\newpage

\bibitem{kerstenprogram2} {\sc P.H.M. Kersten},\,
{\em Software to Compute Infinitesimal Symmetries of Exterior Differential
Systems, with Applications},\,
Acta Appl. Math. 16 (1989) 207.

\bibitem{eliseevprogram} {\sc V.P. Eliseev, R.N. Fedorova} and
{\sc V.V. Kornyak},\, Comput. Phys. Comm. 36 (1985) 383.

\bibitem{fedorovakornyakprogram1} {\sc R.N. Fedorova} and
{\sc V.V. Kornyak},\,
{\em A REDUCE Program for Computing Determining Equations of 
Lie-B\"{a}cklund Symmetries of Differential Equations}.\,
Report R11-87-19 (JINR, Dubna, 1987).

\bibitem{fedorovakornyakprogram2} {\sc R.N. Fedorova} and
{\sc V.V. Kornyak},\, Comput. Phys. Comm. 39 (1986) 93.

\bibitem{fushchichkornyakprogram} {\sc W.I. Fushchich} and
{\sc V.V. Kornyak},\, J. Symb. Comp. 7 (1989) 611.

\bibitem{wolfprogram} {\sc T. Wolf},\,
{\em Analytic solutions of differential equations with computer algebra
systems}.\,
Preprint 87/5 (Universit\"{a}t Jena, Jena, Germany, 1987).

\bibitem{headprogram} {\sc A.K. Head},\, {\em LIE : A muMATH Program
for the Calculation of the LIE Algebra of Differential Equations}\,
(CSIRO Division of Material Sciences, Clayton, Australia, 1990).

\bibitem{vafeadesprogram1} {\sc P. Vafeades},\, {\em Computer Algebraic
Determination of Symmetries and Conservation Laws of PDEs},\,
in: E.K. Park,\, {\em Proc. ISMM Int. Symposium on Computer Applications 
in Design, Simulation and Analysis},\,
(New Orleans, Louisiana, 1990)\, p. 310.

\bibitem{vafeadesprogram2} {\sc P. Vafeades},\, {\em SYMCON: A MACSYMA
Package for the Determination of Symmetries and Conservation Laws of PDEs},\,
submitted to J. Symb. Comp. 8 (1990)

\bibitem{popovprogram} {\sc M.D. Popov},\,
Izvestiya AN Belor. SSR, Ser. Phys.-Math. 2 (1985) 33.

\bibitem{bocharovprogram} {\sc A.V. Bocharov},\,
{\em DEliA: A System of Exact Analysis of Differential Equations using
S. Lie Approach}.\, Report OWIMEX (Program Systems Institute, 
Pereslavl-Zalessky, USSR, 1989).

\bibitem{bocharovbronsteinprogram} {\sc A.V. Bocharov} and 
{\sc M.L. Bronstein}, Acta Appl. Math. 16 (1989) 143.

\bibitem{wulfmanprogram1} {\sc D.K. Davison}, 
{\em MANDO: A Computer Program for Symbolic Manipulation of Differential
Operators Generating Continuous Transformations}, 
M.Sc. Thesis (University of the Pacific, Stockton, CA, 1973).

\newpage

\bibitem{wulfmanprogram2} {\sc G.G. Nagao}, 
{\em DETERMININGEQS: A Computer Program for Aprroximating Lie Generators 
Admitted by Dynamcical Systems}, 
M.Sc. Thesis (University of the Pacific, Stockton, CA, 1980).

\bibitem{rosenauschwarzmeierprogram} {\sc P. Rosenau} and 
{\sc J.L. Schwarzmeier},\, {\em Similarity Solutions of Systems of
Partial Differential Equations using MACSYMA}.\, 
Report COO-3077-160 \, MF-94 (Courant Institute of
Mathematical Sciences, New York University, New York, 1979).

\bibitem{schwarzmeierrosenauprogram} {\sc J.L. Schwarzmeier} and
{\sc P. Rosenau},\,
{\em Using MACSYMA to Calculate Similarity Transformations of Partial
Differential Equations}.\, Report LA-UR \, 88-4157 (Los Alamos 
National Laboratory, Los Alamos, 1988).

\bibitem{steinbergprogram1} {\sc S. Steinberg},\,
in: V. E. Lewis,\, {\em Proc. of the 1979 MACSYMA User's Conference} \, 
(MIT Press, Boston, 1979)\, p. 408

\bibitem{steinbergprogram2} {\sc S. Steinberg},\,
MACSYMA Newsletter 7 (1990) 3. 

\bibitem{champagneprogram} {\sc B. Champagne} and {\sc P. Winternitz}, \,
{\em A MACSYMA Program for Calculating the Symmetry Group of a System
of Differential Equations}.\, Report CRM-1278 (Centre de Recherches 
Math\'{e}matiques, Montr\'{e}al, Canada, 1985).

\bibitem{rosencransprogram} {\sc S.I. Rosencrans},\,
Comput. Phys. Comm. 38 (1985) 347.

\bibitem{schwarzsizeprogram} {\sc F. Schwarz}, 
{\em An Algorithm for Determining the Size of Symmetry Groups}, \,
private communication (1989).

\bibitem{reidprogram1} {\sc G.J. Reid}, in: V. Hussin,
{\em Lie Theory, Differential Equations and Representation Theory}\,
Proc. Annual Seminar of the Canadian Math. Soc. \,
(Les Publications de Centre de Recherches Math\'{e}matiques, Montr\'{e}al, 
Canada, 1990)\, p. 363.

\bibitem{reidprogram2} {\sc G.J. Reid},
{\em Finding symmetries of differential equations without integrating
determining equations}, Technical Report 90-4, Inst. of Appl. Math.,
The University of British Columbia, Vancouver, Canada, submitted to 
European J. Appl. Math (1990).

\bibitem{reidprogram3} {\sc G.J. Reid},
{\em A Triangular Algorithm which Determines the Lie Symmetry Algebra of any
System of PDEs}, J. Phys. A: Math. Gen. 23 (1990) to appear.

\bibitem{karpman1} {\sc V.I. Karpman},\,
Phys. Lett. A 136 (1989) 216.

\end{thebibliography}

% The following pages were used to generated numbered pages for the addenda
% the actual contents of these pages has to be printed separately because
% the files are in ASCII code and not converted into LATEX.

\end{document}
\end
\bye

\newpage % 29

\section*{Test Run Output}

\vfill
\newpage % 30
.
\vfill
\newpage % 31
.
\vfill
\newpage % 32
.
\vfill
\newpage % 33
.
\vfill
\newpage % 34
.
\vfill
\newpage % 35

\section*{Listing of symmgrp.max}

\vfill
\newpage % 36
.
\vfill
\newpage % 37
.
\vfill
\newpage % 38
.
\vfill
\newpage % 39
.
\vfill
\newpage % 40
.
\vfill
\newpage % 41
.
\vfill
\newpage % 42
.
\vfill
\newpage % 43
.
\vfill
\newpage % 44
.
\vfill
\newpage % 45
.
\vfill
\newpage % 46
.
\vfill
\newpage % 47

\section*{Listing of printeqn.max}

\vfill

\end{document}
\end
\bye


